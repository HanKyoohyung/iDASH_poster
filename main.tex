\documentclass[final]{beamer}
\usepackage[scale=1.24]{beamerposter} % Use the beamerposter package for laying out the poster
\usetheme{confposter} % Use the confposter theme supplied with this template
\setbeamercolor{block title}{fg=ngreen,bg=white} % Colors of the block titles
\setbeamercolor{block body}{fg=black,bg=white} % Colors of the body of blocks
\setbeamercolor{block alerted title}{fg=white,bg=dblue!70} % Colors of the highlighted block titles
\setbeamercolor{block alerted body}{fg=black,bg=dblue!10} % Colors of the body of highlighted blocks
% Many more colors are available for use in beamerthemeconfposter.sty

%-----------------------------------------------------------
% Define the column widths and overall poster size
% To set effective sepwid, onecolwid and twocolwid values, first choose how many columns you want and how much separation you want between columns
% In this template, the separation width chosen is 0.024 of the paper width and a 4-column layout
% onecolwid should therefore be (1-(# of columns+1)*sepwid)/# of columns e.g. (1-(4+1)*0.024)/4 = 0.22
% Set twocolwid to be (2*onecolwid)+sepwid = 0.464
% Set threecolwid to be (3*onecolwid)+2*sepwid = 0.708

\newlength{\sepwid}
\newlength{\onecolwid}
\newlength{\twocolwid}
\newlength{\threecolwid}
\setlength{\paperwidth}{48in} % A0 width: 46.8in
\setlength{\paperheight}{36in} % A0 height: 33.1in
\setlength{\sepwid}{0.024\paperwidth} % Separation width (white space) between columns
\setlength{\onecolwid}{0.22\paperwidth} % Width of one column
\setlength{\twocolwid}{0.464\paperwidth} % Width of two columns
\setlength{\threecolwid}{0.708\paperwidth} % Width of three columns
\setlength{\topmargin}{-0.5in} % Reduce the top margin size
%-----------------------------------------------------------

\usepackage{graphicx}  % Required for including images
\usepackage{booktabs} % Top and bottom rules for tables

%----------------------------------------------------------------------------------------
%	TITLE SECTION 
%----------------------------------------------------------------------------------------

\title{Homomorphic Machine Learning with \textsf{SEAL}} % Poster title

\author{Authors} % Author(s)

\institute{Microsoft Research} % Institution(s)

%----------------------------------------------------------------------------------------

\begin{document}

\addtobeamertemplate{block end}{}{\vspace*{2ex}} % White space under blocks
\addtobeamertemplate{block alerted end}{}{\vspace*{2ex}} % White space under highlighted (alert) blocks

\setlength{\belowcaptionskip}{2ex} % White space under figures
\setlength\belowdisplayshortskip{2ex} % White space under equations

\begin{frame}[t] % The whole poster is enclosed in one beamer frame

\begin{columns}[t] % The whole poster consists of three major columns, the second of which is split into two columns twice - the [t] option aligns each column's content to the top

%---------------------------------------------------------------------------------------- COL 1
\begin{column}{\sepwid}\end{column} % Empty spacer column

\begin{column}{\onecolwid} % The first column

% \begin{alertblock}{Objectives}

%\end{alertblock}

%----------------------------------------------------------------------------------------
%	Homomorphic Floor
%----------------------------------------------------------------------------------------

\begin{block}{Homomorphic Floor for SEAL}
\end{block}

\begin{alertblock}{Simple Encrypted Arithmetic Library (SEAL)}
	\begin{itemize}
		\item Homomorphic Encryption library developed by MSR.
		\item This library is based on FV scheme.
	\end{itemize}
\end{alertblock}
%------------------------------------------------

%----------------------------------------------------------------------------------------

\end{column} % End of the first column

\begin{column}{\sepwid}\end{column} % Empty spacer column

\begin{column}{\onecolwid} % The first column

%----------------------------------------------------------------------------------------
%	MATERIALS
%----------------------------------------------------------------------------------------

\begin{block}{1-bit SGD Machine Learning}

\end{block}

%----------------------------------------------------------------------------------------

\end{column} % End of column 2.2

%----------------------------------------------------------------------------------------
%	IMPORTANT To REMEMBER
%----------------------------------------------------------------------------------------

% \begin{alertblock}{Myth of Delta $\Delta$}

% It's commonly believed that in order to work out roots of a quadratic function you must count $\Delta$ and use other previously established formulas. However this is untrue since factorising in many cases is as good or even better than simply counting $\Delta$.

% \end{alertblock} 

%----------------------------------------------------------------------------------------

\begin{column}{\sepwid}\end{column} % Empty spacer column
\begin{column}{\onecolwid} % The second column within column 2 (column 2.2)

%----------------------------------------------------------------------------------------
%	PROOF OF VIETA'S FORMULAS
%----------------------------------------------------------------------------------------

\begin{block}{ Vertical vs. Horizontal Encoding}

\end{block}

%----------------------------------------------------------------------------------------

\end{column} % End of the second column

\begin{column}{\sepwid}\end{column} % Empty spacer column

\begin{column}{\onecolwid} % The third column

%----------------------------------------------------------------------------------------
%	CONCLUSION
%----------------------------------------------------------------------------------------
\setbeamercolor{block title}{fg=red,bg=white} % Change the block title color

\begin{block}{Result}
\begin{itemize}
	\item Time result
    \item Learning time increase linearly to iteration number, we used boostrapping at the end of every iteration.
\end{itemize}
\end{block}

\begin{block}{Further Works}

\end{block}

\end{column}

\end{columns} % End of all the columns in the poster

\end{frame} % End of the enclosing frame

\end{document}
